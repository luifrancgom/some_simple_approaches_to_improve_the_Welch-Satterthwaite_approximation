% Options for packages loaded elsewhere
\PassOptionsToPackage{unicode}{hyperref}
\PassOptionsToPackage{hyphens}{url}
\PassOptionsToPackage{dvipsnames,svgnames,x11names}{xcolor}
%
\documentclass[
  authoryear,
  preprint,
  3p]{elsarticle}

\usepackage{amsmath,amssymb}
\usepackage{iftex}
\ifPDFTeX
  \usepackage[T1]{fontenc}
  \usepackage[utf8]{inputenc}
  \usepackage{textcomp} % provide euro and other symbols
\else % if luatex or xetex
  \usepackage{unicode-math}
  \defaultfontfeatures{Scale=MatchLowercase}
  \defaultfontfeatures[\rmfamily]{Ligatures=TeX,Scale=1}
\fi
\usepackage{lmodern}
\ifPDFTeX\else  
    % xetex/luatex font selection
\fi
% Use upquote if available, for straight quotes in verbatim environments
\IfFileExists{upquote.sty}{\usepackage{upquote}}{}
\IfFileExists{microtype.sty}{% use microtype if available
  \usepackage[]{microtype}
  \UseMicrotypeSet[protrusion]{basicmath} % disable protrusion for tt fonts
}{}
\makeatletter
\@ifundefined{KOMAClassName}{% if non-KOMA class
  \IfFileExists{parskip.sty}{%
    \usepackage{parskip}
  }{% else
    \setlength{\parindent}{0pt}
    \setlength{\parskip}{6pt plus 2pt minus 1pt}}
}{% if KOMA class
  \KOMAoptions{parskip=half}}
\makeatother
\usepackage{xcolor}
\setlength{\emergencystretch}{3em} % prevent overfull lines
\setcounter{secnumdepth}{5}
% Make \paragraph and \subparagraph free-standing
\makeatletter
\ifx\paragraph\undefined\else
  \let\oldparagraph\paragraph
  \renewcommand{\paragraph}{
    \@ifstar
      \xxxParagraphStar
      \xxxParagraphNoStar
  }
  \newcommand{\xxxParagraphStar}[1]{\oldparagraph*{#1}\mbox{}}
  \newcommand{\xxxParagraphNoStar}[1]{\oldparagraph{#1}\mbox{}}
\fi
\ifx\subparagraph\undefined\else
  \let\oldsubparagraph\subparagraph
  \renewcommand{\subparagraph}{
    \@ifstar
      \xxxSubParagraphStar
      \xxxSubParagraphNoStar
  }
  \newcommand{\xxxSubParagraphStar}[1]{\oldsubparagraph*{#1}\mbox{}}
  \newcommand{\xxxSubParagraphNoStar}[1]{\oldsubparagraph{#1}\mbox{}}
\fi
\makeatother


\providecommand{\tightlist}{%
  \setlength{\itemsep}{0pt}\setlength{\parskip}{0pt}}\usepackage{longtable,booktabs,array}
\usepackage{calc} % for calculating minipage widths
% Correct order of tables after \paragraph or \subparagraph
\usepackage{etoolbox}
\makeatletter
\patchcmd\longtable{\par}{\if@noskipsec\mbox{}\fi\par}{}{}
\makeatother
% Allow footnotes in longtable head/foot
\IfFileExists{footnotehyper.sty}{\usepackage{footnotehyper}}{\usepackage{footnote}}
\makesavenoteenv{longtable}
\usepackage{graphicx}
\makeatletter
\def\maxwidth{\ifdim\Gin@nat@width>\linewidth\linewidth\else\Gin@nat@width\fi}
\def\maxheight{\ifdim\Gin@nat@height>\textheight\textheight\else\Gin@nat@height\fi}
\makeatother
% Scale images if necessary, so that they will not overflow the page
% margins by default, and it is still possible to overwrite the defaults
% using explicit options in \includegraphics[width, height, ...]{}
\setkeys{Gin}{width=\maxwidth,height=\maxheight,keepaspectratio}
% Set default figure placement to htbp
\makeatletter
\def\fps@figure{htbp}
\makeatother

\usepackage{booktabs}
\usepackage{longtable}
\usepackage{array}
\usepackage{multirow}
\usepackage{wrapfig}
\usepackage{float}
\usepackage{colortbl}
\usepackage{pdflscape}
\usepackage{tabu}
\usepackage{threeparttable}
\usepackage{threeparttablex}
\usepackage[normalem]{ulem}
\usepackage{makecell}
\usepackage{xcolor}
\makeatletter
\@ifpackageloaded{caption}{}{\usepackage{caption}}
\AtBeginDocument{%
\ifdefined\contentsname
  \renewcommand*\contentsname{Table of contents}
\else
  \newcommand\contentsname{Table of contents}
\fi
\ifdefined\listfigurename
  \renewcommand*\listfigurename{List of Figures}
\else
  \newcommand\listfigurename{List of Figures}
\fi
\ifdefined\listtablename
  \renewcommand*\listtablename{List of Tables}
\else
  \newcommand\listtablename{List of Tables}
\fi
\ifdefined\figurename
  \renewcommand*\figurename{Figure}
\else
  \newcommand\figurename{Figure}
\fi
\ifdefined\tablename
  \renewcommand*\tablename{Table}
\else
  \newcommand\tablename{Table}
\fi
}
\@ifpackageloaded{float}{}{\usepackage{float}}
\floatstyle{ruled}
\@ifundefined{c@chapter}{\newfloat{codelisting}{h}{lop}}{\newfloat{codelisting}{h}{lop}[chapter]}
\floatname{codelisting}{Listing}
\newcommand*\listoflistings{\listof{codelisting}{List of Listings}}
\makeatother
\makeatletter
\makeatother
\makeatletter
\@ifpackageloaded{caption}{}{\usepackage{caption}}
\@ifpackageloaded{subcaption}{}{\usepackage{subcaption}}
\makeatother
\journal{Journal Name}
\ifLuaTeX
  \usepackage{selnolig}  % disable illegal ligatures
\fi
\usepackage[]{natbib}
\bibliographystyle{elsarticle-harv}
\usepackage{bookmark}

\IfFileExists{xurl.sty}{\usepackage{xurl}}{} % add URL line breaks if available
\urlstyle{same} % disable monospaced font for URLs
\hypersetup{
  pdftitle={Some simple approaches to improve the Welch- Satterthwaite approximation},
  pdfauthor={Carlos Alberto Cardozo Delgado; Luis Francisco Gómez López},
  pdfkeywords={Asymmetric distributions, Unbiased estimation, Consistent
estimation, Monte Carlo simulation},
  colorlinks=true,
  linkcolor={blue},
  filecolor={Maroon},
  citecolor={Blue},
  urlcolor={Blue},
  pdfcreator={LaTeX via pandoc}}

\setlength{\parindent}{6pt}
\begin{document}

\begin{frontmatter}
\title{Some simple approaches to improve the Welch- Satterthwaite
approximation}
\author[1]{Carlos Alberto Cardozo Delgado%
\corref{cor1}%
\fnref{fn1}}
 \ead{carlosacardozod@javeriana.edu.co} 
\author[2]{Luis Francisco Gómez López%
\corref{cor1}%
\fnref{fn2}}
 \ead{luifrancgom@gmail.com} 

\affiliation[1]{organization={Pontificia Universidad
Javeriana, Departamento de Matemáticas},addressline={Carrera 7 \# 40 -
62},city={Bogotá},country={Colombia},countrysep={,},postcodesep={}}
\affiliation[2]{organization={Universidad Militar Nueva
Granada, Administración de Empresas},addressline={Kilómetro 2, vía
Cajicá-Zipaquirá},city={Cajicá},country={Colombia},countrysep={,},postcodesep={}}

\cortext[cor1]{Corresponding author}
\fntext[fn1]{This is the first author footnote.}
\fntext[fn2]{This is the first author footnote.}
        
\begin{abstract}
In this work we propose three simple ways to improve the classical
Welch-Satterthwaite (WS) approximation to the effective degree of
freedom of a non-negative linear combination of \(\chi^2\)
distributions. The WS option is typically used in design of experiments
and metrology. However, it has been pointing out in many references the
multiple limitations of the inferences based on the WS approximation.
Three novel estimators of the effective degree of freedom of a
non-negative linear combination of \(\chi^2\) distributions are given.
We also study some theoretical properties of the proposed estimators.
Additionally, through several Monte Carlo simulations, we assess the
bias, variance, and mean square error of the proposed estimators under
(very) small and moderate sample sizes. The proposed estimators have a
much better performance than the WS proposal and his implications.
Finally, two applications are presented in which the proposed estimators
help to improve the performance of some interval estimation and
hypothesis testing procedures.
\end{abstract}





\begin{keyword}
    Asymmetric distributions \sep Unbiased estimation \sep Consistent
estimation \sep 
    Monte Carlo simulation
\end{keyword}
\end{frontmatter}
    
\section{Critical aspects of the WS
approach}\label{critical-aspects-of-the-ws-approach}

\citet{lloyd_2-sample_2013} proposed basically three numerical
experiments that pointed out some weak points of the two-sample
t-distribution test based on the WS approach. He assumed, in his first
experiment, that the exact proportion between the variances of two
population is known. In our view, it is rare in practice to known it.
So, we will remake that simulation but we are not assuming that the
variance proportion is known because we believe that it gives us a more
realistic idea of the performance of the two-sample t-distribution test
under the WS approximation.

Now, we describe the plan of this initial simulation. We use the
following proportions between the variances, \(\rho = 2,4,6,8,10\).
Also, we will assume that the two samples have the same size, \(n\), and
generate random samples of sizes \(n = 2,3,\cdots,12\). In the next
step, for each pair of \(\rho\) and \(n\), we will build the sampling
distribution of the two-sample t-test approximation based on \(R=10000\)
replications but we are not using a previously calculated and fixed
value of \(\nu\) given by the WS approach as \citet{lloyd_2-sample_2013}
did it. Finally, we calculate the empirical proportion of \ldots{}

\[R = 10000, \alpha = 0.1 \text{ and } 0.05\]

\[m_1,m_2 , \text{based on a fixed n=6 and } \frac{\sigma_2^2}{\sigma_1^2} = 2\]

\[t_{0.05,\hat{\nu}^j}, t_{0.95,\hat{\nu}^j},\text{where } \hat{\nu}^j \text{ is d.f calculated by the WS approach.}\]

\[F_{T}: \text{\ Empirical Cumulative Distribution of the t-test}\]

\[\hat{\alpha}_j = F_{T}(t_{0.05,\hat{\nu}^j}) + ( 1 - F_{T}( t_{0.95,\hat{\nu}^j})), \text{\ Type I Error based on WS approach}\]

We want to assess the proportion of \(\hat{\alpha}_j\) that are smaller
or bigger than \(\alpha\) in \(d=0.1\) in a relative way:

\[\frac{|\hat{\alpha}_j - \alpha|}{\alpha} > 0.1\]

\begin{table}

\caption{\label{tbl-simulations-alpha}Proportion of replicates where the
relative absolute difference between a nominal \(\alpha\) and the
\(\alpha\)-value given by WS approximation test is bigger than 0.1 where
\(\alpha = 0.1, 0.05.\)}

\centering{

\centering
\begin{tabular}[t]{rrlllllllllll}
\toprule
\multicolumn{2}{c}{ } & \multicolumn{11}{c}{$n$} \\
\cmidrule(l{3pt}r{3pt}){3-13}
$\alpha$ & $\rho$ & 2 & 3 & 4 & 5 & 6 & 7 & 8 & 9 & 10 & 11 & 12\\
\midrule
 & 2 & \cellcolor[HTML]{ccbe93}{0.306} & \cellcolor[HTML]{ccbe93}{0.208} & \cellcolor{white}{0.02} & \cellcolor{white}{0.038} & \cellcolor[HTML]{ccbe93}{0.138} & \cellcolor[HTML]{ccbe93}{0.228} & \cellcolor[HTML]{ccbe93}{0.198} & \cellcolor[HTML]{ccbe93}{0.338} & \cellcolor[HTML]{ccbe93}{0.304} & \cellcolor[HTML]{ccbe93}{0.318} & \cellcolor[HTML]{ccbe93}{0.416}\\
\cmidrule{2-13}
 & 4 & \cellcolor{white}{0} & \cellcolor{white}{0.084} & \cellcolor[HTML]{ccbe93}{0.258} & \cellcolor[HTML]{ccbe93}{0.384} & \cellcolor[HTML]{ccbe93}{0.49} & \cellcolor[HTML]{ccbe93}{0.54} & \cellcolor[HTML]{ccbe93}{0.624} & \cellcolor[HTML]{ccbe93}{0.686} & \cellcolor[HTML]{ccbe93}{0.684} & \cellcolor[HTML]{ccbe93}{0.762} & \cellcolor[HTML]{ccbe93}{0.792}\\
\cmidrule{2-13}
 & 6 & \cellcolor[HTML]{ccbe93}{0.118} & \cellcolor[HTML]{ccbe93}{0.162} & \cellcolor[HTML]{ccbe93}{0.376} & \cellcolor[HTML]{ccbe93}{0.58} & \cellcolor[HTML]{ccbe93}{0.632} & \cellcolor[HTML]{ccbe93}{0.66} & \cellcolor[HTML]{ccbe93}{0.798} & \cellcolor[HTML]{ccbe93}{0.78} & \cellcolor[HTML]{ccbe93}{0.854} & \cellcolor[HTML]{ccbe93}{0.884} & \cellcolor[HTML]{ccbe93}{0.904}\\
\cmidrule{2-13}
 & 8 & \cellcolor[HTML]{ccbe93}{0.268} & \cellcolor[HTML]{ccbe93}{0.29} & \cellcolor[HTML]{ccbe93}{0.442} & \cellcolor[HTML]{ccbe93}{0.586} & \cellcolor[HTML]{ccbe93}{0.716} & \cellcolor[HTML]{ccbe93}{0.83} & \cellcolor[HTML]{ccbe93}{0.852} & \cellcolor[HTML]{ccbe93}{0.848} & \cellcolor[HTML]{ccbe93}{0.892} & \cellcolor[HTML]{ccbe93}{0.916} & \cellcolor[HTML]{ccbe93}{0.926}\\
\cmidrule{2-13}
\multirow{-5}{*}{\raggedleft\arraybackslash 0.05} & 10 & \cellcolor[HTML]{ccbe93}{0.34} & \cellcolor[HTML]{ccbe93}{0.336} & \cellcolor[HTML]{ccbe93}{0.5} & \cellcolor[HTML]{ccbe93}{0.656} & \cellcolor[HTML]{ccbe93}{0.768} & \cellcolor[HTML]{ccbe93}{0.848} & \cellcolor[HTML]{ccbe93}{0.86} & \cellcolor[HTML]{ccbe93}{0.916} & \cellcolor[HTML]{ccbe93}{0.932} & \cellcolor[HTML]{ccbe93}{0.93} & \cellcolor[HTML]{ccbe93}{0.966}\\
\cmidrule{1-13}
 & 2 & \cellcolor[HTML]{ccbe93}{0.165} & \cellcolor{white}{0.052} & \cellcolor{white}{0.088} & \cellcolor[HTML]{ccbe93}{0.131} & \cellcolor[HTML]{ccbe93}{0.228} & \cellcolor[HTML]{ccbe93}{0.258} & \cellcolor[HTML]{ccbe93}{0.282} & \cellcolor[HTML]{ccbe93}{0.378} & \cellcolor[HTML]{ccbe93}{0.343} & \cellcolor[HTML]{ccbe93}{0.369} & \cellcolor[HTML]{ccbe93}{0.414}\\
\cmidrule{2-13}
 & 4 & \cellcolor{white}{0.048} & \cellcolor[HTML]{ccbe93}{0.188} & \cellcolor[HTML]{ccbe93}{0.325} & \cellcolor[HTML]{ccbe93}{0.459} & \cellcolor[HTML]{ccbe93}{0.513} & \cellcolor[HTML]{ccbe93}{0.566} & \cellcolor[HTML]{ccbe93}{0.646} & \cellcolor[HTML]{ccbe93}{0.669} & \cellcolor[HTML]{ccbe93}{0.691} & \cellcolor[HTML]{ccbe93}{0.744} & \cellcolor[HTML]{ccbe93}{0.753}\\
\cmidrule{2-13}
 & 6 & \cellcolor[HTML]{ccbe93}{0.11} & \cellcolor[HTML]{ccbe93}{0.225} & \cellcolor[HTML]{ccbe93}{0.476} & \cellcolor[HTML]{ccbe93}{0.621} & \cellcolor[HTML]{ccbe93}{0.678} & \cellcolor[HTML]{ccbe93}{0.703} & \cellcolor[HTML]{ccbe93}{0.786} & \cellcolor[HTML]{ccbe93}{0.785} & \cellcolor[HTML]{ccbe93}{0.846} & \cellcolor[HTML]{ccbe93}{0.871} & \cellcolor[HTML]{ccbe93}{0.88}\\
\cmidrule{2-13}
 & 8 & \cellcolor[HTML]{ccbe93}{0.169} & \cellcolor[HTML]{ccbe93}{0.403} & \cellcolor[HTML]{ccbe93}{0.521} & \cellcolor[HTML]{ccbe93}{0.643} & \cellcolor[HTML]{ccbe93}{0.742} & \cellcolor[HTML]{ccbe93}{0.832} & \cellcolor[HTML]{ccbe93}{0.845} & \cellcolor[HTML]{ccbe93}{0.865} & \cellcolor[HTML]{ccbe93}{0.89} & \cellcolor[HTML]{ccbe93}{0.909} & \cellcolor[HTML]{ccbe93}{0.908}\\
\cmidrule{2-13}
\multirow{-5}{*}{\raggedleft\arraybackslash 0.10} & 10 & \cellcolor[HTML]{ccbe93}{0.261} & \cellcolor[HTML]{ccbe93}{0.413} & \cellcolor[HTML]{ccbe93}{0.58} & \cellcolor[HTML]{ccbe93}{0.721} & \cellcolor[HTML]{ccbe93}{0.792} & \cellcolor[HTML]{ccbe93}{0.864} & \cellcolor[HTML]{ccbe93}{0.869} & \cellcolor[HTML]{ccbe93}{0.916} & \cellcolor[HTML]{ccbe93}{0.935} & \cellcolor[HTML]{ccbe93}{0.928} & \cellcolor[HTML]{ccbe93}{0.957}\\
\bottomrule
\end{tabular}

}

\end{table}%


\renewcommand\refname{References}
  \bibliography{bibliography.bib}


\end{document}
